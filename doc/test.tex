\documentclass{report}
\usepackage[OT4]{polski}
\usepackage[utf8]{inputenc} 
\usepackage{hyperref}
\usepackage{fancyhdr}
\pagestyle{fancy}
\lhead{\textbf{ZPR}}
\chead{}
\rhead{\bfseries Dokumentacja wstępna projektu.}
\lfoot{Jacek Sosnowski, Łukasz Gadawski}
\cfoot{}
\rfoot{\thepage}
\renewcommand{\headrulewidth}{0.4pt}
\renewcommand{\footrulewidth}{0.4pt}

\author{Jacek Sosnowski, Łukasz Gadawski}
\title{Dokumentacja wstępna projektu: \\
\textbf{zadanie 1 - gra z interfejsem na przeglądarkę}}
\begin{document}
\maketitle

\begin{enumerate}
\item \textbf{Opis projektu} \\
Projekt będzie realizował grę sieciową dostępną dla użytkownika z poziomu przeglądarki.
Planowaną grą jest kółko i krzyżyk na zasadach opisanych na \url{http://pl.wikipedia.org/wiki/Kólko_i_krzyżyk} .
Wybrana została klasyczna odmiana bez rozszerzeń na planszy 3x3. Aplikacja będzie zrealizowana w architekturze klient-serwer.

\item \textbf{Szczegółowy opis gry} \\
Użytkownik będzie rozpoczynał grę poprzez podanie swojego identyfikatora. Następnie wyświetlona zostanie lista osób, 
które są dostępne na serwerze. Po czym będzie mógł wysłać zaproszenie do rozgrywki dla wybranej osoby. Akceptacja
zaproszenia spowoduje rozpoczęcie rywalizacji. \\
Użytkownik za wygraną partie otrzyma 3 punkty, za remis 1, natomiast za przegraną nie otrzyma żadnych punktów. 
Dorobek gracza będzie przechowywany na liście rankingowej znajdującej się na serwerze. 

\item \textbf{ Założenia projektowe}	\\
Aplikacja będzie:
\begin{enumerate}
\item wykorzystywała przeglądarkę jako interfejs graficzny (realizacja koncepcji cienkiego klienta)
\item uruchamiana na serwerze http dostarczanym przez bibliotekę WT
\item posiadała interfejs formatowany poprzez style CSS
\item wykorzystywała wielowątkowość
\item przenośna między różnymi platformami
\item funkcjonowała w sieci lokalnej lub w sieci internet pod warunkiem dostępności publicznego adresu IP dla serwera
\item wykorzystywała IP wersji 4
\end{enumerate}

\item \textbf{Wykorzystywane biblioteki}
\begin{enumerate}
\item Boost 
\item WT(WebToolkit) \url{http://www.webtoolkit.eu/wt}
\item STL
\end{enumerate}

\item \textbf{Testowanie}
Planowane jest wykorzystanie testów jednostkowych z biblioteki boost. 
\end{enumerate}

\end{document}
